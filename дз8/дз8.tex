\documentclass[11pt]{article}
\usepackage{preamble}
\usepackage{gset}
\def\week{8}
\def\FN{\mathcal{N}_f}
\begin{document}
	\setcounter{problem}{0}
	\def\theproblem{Д\week.\arabic{problem}}
	{\textbf{\large Дискретная математика}\hfill \textbf{(Основной поток)}
		
		\medskip %
		
		\textbf{Домашнее задание \week}}
	
	\medskip
	
	\textbf{Дайте обоснованные ответы на следующие вопросы.}
	
	
	\vspace{5mm}
	
	\p \sp Верно ли, что если $|A\setminus B|=|B\setminus A|$, то $|A|=|B|$?
	
	$|A|= |A \cup B| + |A \setminus B|, |B| = |A \cup B| + |B \setminus A| \Rightarrow |A| = |B|$ \\
	\answer{Да}
	
	\sp Верно ли, что если $|A|=|C|$, $|B|=|D|$, $B\subseteq A$, $D\subseteq C$, то $|A\setminus B|=|C\setminus D|$?
	
	$|A| = |B| + |A \setminus B|, |C| = |A| = |D| + |C \setminus D| \Rightarrow |B| + |A \setminus B| = |D| + |C \setminus D| \Rightarrow |A \setminus B| = |C \setminus D|$
	
	\answer{Да}
	
	\p Докажите, что множество конечных подмножеств рациональных чисел
	счётно.
	
	Для каждого подмножества рациональных чисел можно найти рациональное число, которое будет соответствовать произведению элементов подмножества. Далее можно выписать все полученные произведения в порядке неубывания, поэтому 
	
	\p Функция периодическая, если для
	некоторого числа $T$ и любого $x$ выполняется $f(x+T)=f(x)$. Докажите,
	что множество периодических функций  $f\colon \Z\to\Z$ счётно.
	
	\p Тотальную функцию из $\NN$ в $\NN$ назовём представительной, если она строго возрастающая и её значения~"--- все натуральные числа за исключением конечного множества. Докажите, что множество представительных  функций счётно.
	
	\p Множество $A$ состоит из бесконечных последовательностей десятичных цифр (то есть,  элементов множества \{0, 1, \ldots, 9\}), в которых цифра 5 встречается на втором месте, а больше эта цифра нигде не встречается. Является ли это множество счётным?
	
	
\end{document}