\documentclass[11pt]{article}
\usepackage{preamble}
\usepackage{gset}
\def\week{8}
\def\FN{\mathcal{N}_f}
\begin{document}
	\setcounter{problem}{0}
	\def\theproblem{Д\week.\arabic{problem}}
	{\textbf{\large Дискретная математика}\hfill \textbf{(Основной поток)}
		
		\medskip %
		
		\textbf{Домашнее задание \week}}
	
	\medskip
	
	\textbf{Дайте обоснованные ответы на следующие вопросы.}
	
	
	\vspace{5mm}
	
	\p \sp Верно ли, что если $|A\setminus B|=|B\setminus A|$, то $|A|=|B|$?
	
	$|A|= |A \cup B| + |A \setminus B|, |B| = |A \cup B| + |B \setminus A| \Rightarrow |A| = |B|$ \\
	\answer{Да}
	
	\sp Верно ли, что если $|A|=|C|$, $|B|=|D|$, $B\subseteq A$, $D\subseteq C$, то $|A\setminus B|=|C\setminus D|$?
	
	$|A| = |B| + |A \setminus B|, |C| = |A| = |D| + |C \setminus D| \Rightarrow |B| + |A \setminus B| = |D| + |C \setminus D| \Rightarrow |A \setminus B| = |C \setminus D|$
	
	\answer{Да}
	
	\p Докажите, что множество конечных подмножеств рациональных чисел
	счётно.
	
	Мы уже знаем, что множество простых чисел счетно, потому что это бесконечное подмножество натуральных чисел. Также мы знаем, что множество рациональных чисел счетно. Это значит, что каждому рациональному числу можно сопоставить простое число. Для каждого конечного подмножества рациональных чисел вычислим произведение простых чисел, которые сопоставлены элементам множества. Получим уникальное натуральное число, которое будет соответствовать подмножеству. Получается, что в итоге множеству всех конечных подмножеств рациональных чисел можно сопоставить счетное подножество натуральных чисел. Значит изначальное множество счетно. 
		
	\p Функция периодическая, если для
	некоторого числа $T$ и любого $x$ выполняется $f(x+T)=f(x)$. Докажите,
	что множество периодических функций  $f\colon \Z\to\Z$ счётно.
	
	Рассмотрим множество всех функций с фиксированным периодом n. Оно счетно, потому что $|T_n|= |Z|^n = |N|^n = |N|.$ Множество всех периодических функций можно задать как объединение $T_1 \cup T_2 \cup \cdots$. То есть как объединение счетных множеств, что значит, что изначальное множество счетно.
	
	\p Тотальную функцию из $\NN$ в $\NN$ назовём представительной, если она строго возрастающая и её значения~"--- все натуральные числа за исключением конечного множества. Докажите, что множество представительных  функций счётно.
	
	Представим функцию как последовательность $a_n = \{a_0, a_0 + d_1, a_0 + d_1 + d_2, \cdots\}$. Так как функция вострастающая, то $d_i  \geq 1, d_i \in \mathbb{N}$. Также заметим, что $d_i \neq 1$, если нужно "перепрыгнуть" \space значение из конечного множетсва, которое исключили, потому что иначе $d_i = 1$. Таких перепрыгиваний конечное число, потому что множество натуральных чисел, которое исключили тоже конечно. Получается, что функцию можно описывать только до $d_{i_{max}}$, то есть до самого большого индекса, который не равен 1, потому что потом все индексы будут равны единице. Получается, что каждую функцию можно описать конечной последовательностью, а значит множество всех функций как объединение конечных множеств счетно. 
		
	\p Множество $A$ состоит из бесконечных последовательностей десятичных цифр (то есть,  элементов множества \{0, 1, \ldots, 9\}), в которых цифра 5 встречается на втором месте, а больше эта цифра нигде не встречается. Является ли это множество счётным?
	
	Пусть множество является счетным. Выпишем последовательности из множества друг под другом. 
	
	$a_0^1, 5, a_2^1, \cdots$
	
	$a_0^2, 5, a_2^2, \cdots$
	
	$a_0^3, 5, a_2^3, \cdots$
	
	$\vdots$
	
	Пусть $\sigma = (2,3,4,6,5,7,8,9,1)$ - перестановка. 
	
	Составим новую последовательность $\sigma(a_0^1), 5, \sigma(a_2^2), \sigma(a_3^3), \cdots$
	
	Новая последовательность отличается от первой первым элементом, со второй третьим, с третьей четвертым и так далее, потому что ни в какой позиции больше не стоит 5, а значит нигде кроме 2 элемента $\sigma(i) \neq i$.
	
	Мы получили новую последовательность, которая не была выписана, то есть получили противоречие. А значит, множество не является счетным. 
\end{document}