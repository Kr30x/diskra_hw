\documentclass[11pt]{article}
\usepackage{preamble}
\usepackage{gset}
\def\week{2}
\def\theproblem{К\week.\arabic{problem}}
\begin{document}
	\setcounter{problem}{0}
	\def\theproblem{Д\week.\arabic{problem}}
	{\textbf{\large Дискретная математика}\hfill \textbf{(Основной поток)}
		
		\medskip %
		
		\textbf{Домашнее задание \week}}
	
	\medskip
	
	\textbf{Дайте обоснованные ответы на следующие вопросы.}
	
	
	\vspace{5mm}
	
	
	\p Найдите количество нулей в последовательности $B$, определённой в задаче К2.1.\\
	Последовательность: \[
	(1,\;2,\;3,\;4,\;5,\;6,\;7,\;8,\;9,\;1,\;0,\;1,\;1,\;1,\;2,\;\dots )
	\]
	Среди однозначных чисел >1 нет тех, среди которых есть с цифрой 0.\\
	Среди двузначных чисел цифра ноль есть в числах, которые имеют вид $n = 10k, k < 9, k \in \mathbb{N}$. Таких чисел 9. Соответственно и нулей 9.\\
	Трехзначные числа имеют ноль, либо если они вида $\overline{a0b}$, a - цифра от 1 до 9, b - цифра от 0 до 9. Всего таких чисел $9 * 10 = 90$. Либо если они вида $\overline{ab0}$, а - цифра от 1 до 9, b - цифра от 0 до 9. Всего таких чисел $9 * 10 = 90$. Важно не забыть про числа вида $\overline{a00}$, которые входят в оба множества, поэтому мы их посчитали дважды, но и нуля в них 2, поэтому лишних нулей не будет. 
	Поэтому всего нулей среди чисел промежутка $[100, 999]$ 180.\\
	Остались только цетырехзначные числа... Разберем только числа в промежутке $[1000, 2000)$ С 1 нулем числа вида $\overline{10bc}, \overline{1b0c}, \overline{1bc0}$. Их $ 9 * 9 * 3 = 243$. Числа с нулями могут иметь вид $\overline{100a}, \overline{10a0}, \overline{1a00}$. Их $9 * 3 = 27$. Нулей в них $27 * 2 = 54$. Осталось добавить 3 нуля от числа 1000. Всегда нулей в промежутке $[1000, 2000)$ 300\\
	Остальные числа разберу вручную. 2000 + 3 нуля. 2001 - 2009 + 18 нулей. 2010 + 2 нуля. 2011 - 2019 + 9 нулей. 2020 + 2 нуля. 2021 -2023 + 3 нуля. Итого + 37 нулей.\\
	\\
	Итого: 9 + 180 + 300 + 37 = 526. 
	
	\p Для любого целого положительного $n$ докажите равенство
	\[
	n\cdot2^0+(n-1)\cdot 2^1+(n-2)\cdot2^2+(n-3)\cdot2^3\dots+1\cdot 2^{n-1} = 2^{n+1}-2-n.
	\]
	Докажем методом математической индукции. \\
	\\
	База индукции:\\
	\\
	Для n = 1: $1 \cdot 2 ^ 0 = 2 ^ {1 + 1} - 2 - 1$, верно \\
	\\
	Шаг индукции: \\
	\\
	Левая часть: $(n + 1)\cdot2^0+(n + 1 - 1)\cdot 2^1+(n + 1 - 2)\cdot2^2+(n + 1 - 3)\cdot2^3\dots+1\cdot 2^{n} = 2^{n + 1} - 2 - n + 2 + 2^2 + 2 ^ 3 + \cdots + 2 ^{n} = 2^{n  + 1} - 2 - n + 2^{n + 1} - 1 = 2 ^ {n + 2} - 2 - n - 1$\\
	Так как верна база индукции и ее шаг, то можно сделать вывод, что исходное утверждение верно для любых целых положительных n.
	\\
	 
	\p \sp Докажите, что \[(A_1\sm A_2)\times (B_1\sm B_2)\subseteq (A_1\times B_1)\sm (A_2\times B_2)\] для любых множеств $A_1$, $A_2$, $B_1$, $B_2$.\\
	$
	(A_1\sm A_2)\times (B_1\sm B_2) = \{(x, y): (x \in A_1\land x \notin A_2) \land (y \in B_1 \land y \notin B_2)\} = M\\
	(A_1\times B_1)\sm (A_2\times B_2) = \{(x, y): x \in A_1 \land y \in B_1 \land (x \notin A_2 \lor y \notin B_2)\} = N\\
	$\\
	$(a_1 \land \neg a_2) \land (b_1 \land \neg b_2) \lif (a_1 \land \neg a_2 \land b_1) \lor (a_1 \land b_1 \land \neg b_2)$	\\
	Значит $M \subseteq N$\\
		
	\sp Выполняется ли обратное включение для любых множеств  $A_1$, $A_2$, $B_1$, $B_2$? 
	\\
	Нет. Приведем контрпример. $A_1 = \{1\}; A_2 = \{1\}; B_1 = \{3\}; B_2 = \{4\}$\\
	$(A_1\times B_1)\sm (A_2\times B_2) = \{(1, 3)\}$\\
	$(A_1\sm A_2)\times (B_1\sm B_2) = \varnothing$\\
	 $\{(1, 3)\} \nsubseteq \varnothing$\\
	 
	
	\p В последовательность $A = (x_1,\dots, x_{2n})$ входят целые числа от 1 до $n$, каждое из этих чисел входит в $A$ ровно два раза. Известно, что для любых $1\leq a, b\leq n$, $a\ne b$, после вычёркивания  из $A$ всех чисел за исключением $a$, $b$ получается либо последовательность $(a,b,a,b)$, либо последовательность $(b,a,b,a)$. Докажите, что в последовательности $(x_1,\dots, x_n)$ каждое число от 1 до $n$ встречается ровно один раз.\\
	
	Чтобы доказать, что в последовательности $(x_1,\dots, x_n)$ каждое число от 1 до $n$ встречается ровно один раз, докажем, что если это условие не выполняется, то не выполняется и условие, что для любых $1\leq a, b\leq n$, $a\ne b$, после вычёркивания  из $A$ всех чисел за исключением $a$, $b$ получается либо последовательность $(a,b,a,b)$, либо последовательность $(b,a,b,a)$. Тогда исходное утверждение будет доказано по закону контрапозиции.\\\\
	Пусть в последовательности $B = (x_1,\dots, x_n)$ есть повторяющиеся элементы, соответственно в этой последовательности различных элементов максимум $n - 1$. Тогда в последовательности $С = (x_{n + 1},\dots, x_{2n})$ какой-то элемент встречается 2 раза. Назовем элемент, который повторяется в последовательности B x, а в C y. Тогда после вычеркивания всех остальных элементов кроме x, y останется последовательность вида $(x,x,y,y)$. \\
	\\
	Следовательно  в последовательности $(x_1,\dots, x_n)$ каждое число от 1 до $n$ встречается ровно один раз.
	
	\end{document}