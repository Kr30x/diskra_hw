\documentclass[11pt]{article}
\usepackage{preamble}
\usepackage[]{gset}
\def\week{9}
\def\theproblem{К\week.\arabic{problem}}


\def\FN{\mathcal{N}_f}
\begin{document}
	\setcounter{problem}{0}
	\def\theproblem{Д\week.\arabic{problem}}
	{\textbf{\large Дискретная математика}\hfill \textbf{(Основной поток)}
		
		\medskip %
		
		\textbf{Домашнее задание \week}}
	
	\medskip
	
	\textbf{Дайте обоснованные ответы на следующие вопросы.}
	
	
	\vspace{5mm}
	
	
	\p Рассмотрим бесконечные последовательности из $0$, $1$ и $2$, в
	которых никакая цифра не встречается два раза подряд. Верно ли, что
	мощность множества таких последовательностей имеет мощность континуум?
	
	Представим последовательность $a$ размера $n$ как начальный элемент + последовательность  ${0,1}$ размера $n - 1$ по такому принципу. Пусть мы знаем последовательность до $i$-ого элемента, если $a_i = 0$, в последовательность 0, если $a_{i + 1} = 1$, 1, если $a_{i + 1} = 2$. Аналогично, если $a_i = 1$, запишем 0, если $a_{i + 1} = 0$, 1, если $a_{i + 1} = 2$. Для $a_i = 2$, если $a_{i + 1} = 0$, запишем 0, иначе 1. Получается, что мы построили биекцию из множества заданных последовательностей в множество $\{0,1,2\}\times{\{0,1\}}_{n = 1}^{\infty} \Rightarrow |\{0,1,2\}| = |{\{0,1\}}_{n = 1}^{\infty}| = C$ 
	
	\answer{Да}
	
	\p Рассмотрим множество пар различных действительных чисел, то есть
	\[\bar D = \{(x,y) : x\ne y, \ x,y\in \RR\}.\] Является ли множество $\bar D$ континуальным?
	
	$|D| = |\RR \times \RR|, |\RR \times \RR| = |\RR|$ (из лекции), значит $|D| = |\RR| = C$ 
	
	\answer{Да}
	
	\p Является ли множество всех тотальных функций $\RR\to\RR$ континуальным?
	
	Множество всех тотальных функций $\RR\to\RR$ имеет мощность $\RR^{\RR}$
	
	Мощность всех подмножеств $\RR\to\RR$ имеет мощность $2^\RR$. $\RR^{\RR} > 2^\RR, |2^\RR| > |\RR| = C$ $\Rightarrow |\RR^{\RR}| > C$ 
	
	\answer{Нет}
	
	\p Функция периодическая, если для некоторого числа $T>0$ (периода) и любого $x$ выполняется $f(x+T)=f(x)$. Счётно ли множество множество периодических функций  $f\colon \Q\to\Q$? Период считайте рациональным. 
	
	Можем от каждой функции оставить только ее первый положительный переиод, то есть когда первый аргумент в  периоде положителен. Получим биекцию из множества  $\Q\to\Q$ в множетсво полуотрезков на множестве $\Q$, так как по условию функция тотальная (мы подставляем в функцию любой $x$). Каждый полуотрезок отрезок можно задать парой рациональный чисел - началом и концом. Получаем биекцию в множество $\Q \times \Q$, так как $|\Q| = |\N|$, то можно построить биекцию в множество $\N \times \N$, которое счетно, потому что его можно разбить в в объединение счётного числа счётных множеств $\{0\} \times N, \{1\} \times N, \{2\} \times \N, \cdots$. 
	
	\answer{Да}
	
\end{document}