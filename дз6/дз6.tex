\documentclass[11pt]{article}
\usepackage{preamble}
\usepackage{gset}
\def\week{6}
\def\theproblem{К\week.\arabic{problem}}
\begin{document}
	\setcounter{problem}{0}
	\def\theproblem{Д\week.\arabic{problem}}
	{\textbf{\large Дискретная математика}\hfill \textbf{(Основной поток)}
		
		\medskip %
		
		\textbf{Домашнее задание \week}}
	
	\medskip
	
	\textbf{Дайте обоснованные ответы на следующие вопросы.}
	
	
	\vspace{5mm}
	
	\p Найдите количество сюръективных неубывающих функций из $[10]$ в $[7]$. Функция $f$ неубывающая, если $x\leq y$ влечёт $f(x) \leq f(y)$.
	
	Из сюрьективности следует, что множество $[10]$ можно разбить на ровно 7 неперсекающихся подмножеств, объединяя их по одинаковым значениям после функции. Из неубывания функции можно вывести следствие, что получившиеся подмножества представляют из себя ряды подряд идущих элементов множества $[10]$. Пусть это не так, тогда $n - 1 \in A, n \in B, n + k \in A$. Это противоречит условию неубывания. То есть можно перефразировать задачу в задачу, напоминающую задачу про способы разрубки (если есть такое слово) на n частей. Иными словами нужно расположить 6 распилов (или разделов множества) в 9 позиций. Сделать это можно $C_9^6 = \frac{9!}{6!3!} = \frac{7*8*9}{2 * 3} = 84$. \sspace
	\answer{84} 
	
	\p Докажите, что
	\[
	\sum_{j=0}^k \binom{n+j-1}{j} = \binom{n+k}{k}\,.
	\]
	\bs
	Докажем по индукции по k. \sspace
	\textbf{База k = 1:} $\sum_{j=0}^1 \binom{n+j-1}{j} = \binom{n+1}{1}$ \sspace
	$\binom{n - 1}{0} + \binom{n}{1}  = \binom{n + 1}{1}$ \sspace
	$1 + n = n + 1$ - верно \sspace
	\textbf{Шаг k = k + 1:} $\sum_{j=0}^{k + 1} \binom{n+j-1}{j} = \binom{n+k+1}{k + 1}$ \sspace
	$\sum_{j=0}^{k + 1} \binom{n+j-1}{j} = \sum_{j=0}^k \binom{n+j-1}{j} + \binom{n + k + 1 - 1}{k + 1}= \binom{n+k}{k} + \binom{n + k + 1 - 1}{k + 1} = \binom{n + k}{k} + \binom{n + k}{k + 1} = \dfrac{(n + k)!}{k!(n - k)!} + \dfrac{(n + k)!}{(k + 1)!(n + k - k - 1)!} = \sspace = $ 
	$\dfrac{(n + k)!}{k!(n - k)!} + \dfrac{(n + k)!}{(k + 1)!(n + k - k - 1)!} = \dfrac{(n + k)!(1 + \frac{k + 1}{n})}{(k + 1)!n!} = $
	$\dfrac{(n + k)!(n + k + 1}{(k + 1)!(n - 1)!} = \dfrac{(n + k + 1)!}{(k + 1)!(n - 1)!}$
	\bs
	\p
	Сколько различных слов (не обязательно осмысленных) можно получить,
	переставляя буквы в слове ABRACADABRA так, чтобы никакие
	две буквы A не стояли рядом? Ответом должно быть число в десятичной записи.
	\bs
	Сначала посчитаем слова только из согласных, их $\dfrac{6}{2!2!} = \dfrac{720}{4} = 180$ \\
	Затем расставим буквы А согласно условию, для этого воспользуемся формулой. Для этого выберем 5 мест из 7. Семь потому что букву А можно поставить как внутри слова, так и в его начале или конце. Вот схема слова, где С означает согласную букву: ?C?C?C?C?C?C?. На места вопросов надо поставить буквы А. Сделать это можно $\binom{7}{5}$ способами. Итого 21 способ. Получается что различных слов $21*180 = 3780$. 
	\sspace
	\answer{3780}
	
	\p Сравните числа (равны ли; если нет, то какое больше):
	\[
	\sum_{i=0}^{512} 2^{2i}\binom{1024}{2i} \quad\text{и}\quad
	\sum_{i=0}^{511} 2^{2i+1}\binom{1024}{2i+1}.
	\] 
	$\sum_{i=0}^{512} 2^{2i}\binom{1024}{2i} - \sum_{i=0}^{511} 2^{2i+1}\binom{1024}{2i+1} = \sum_{a = 0}^{1024} (-1)^{a} * 2^a * \binom{1024}{a} $
	\sspace
	Почти можем сложить в бином, не хватает только, чтобы -1 был в степени $1024 - a$. Но мы можем спокойно заменить $a$ на $1024 - a$, потому что мы не поменяем четность, а значит не поменяем и само число -1 в степени.
	\sspace
	Значит теперь получим запись  $\sum_{a = 0}^{1024} (-1)^{1024-a} * 2^a * \binom{1024}{a} = (2 - 1)^1024 = 1$. Так как 1 > 0, значит $\sum_{i=0}^{512} 2^{2i}\binom{1024}{2i}$ больше.  
\end{document}