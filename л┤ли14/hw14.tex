\documentclass[11pt]{article}


\usepackage{gset}
\usepackage{preamble}
\def\week{14}
\def\theproblem{К\week.\arabic{problem}}

\begin{document}


{\textbf{\large Дискретная математика (основной поток)}\hfill
  \textbf{\large Занятие \week}
}

\setcounter{problem}{0}
\def\theproblem{Д\week.\arabic{problem}}
{\textbf{\large Дискретная математика}\hfill \textbf{(Основной поток)}

\medskip %

\textbf{Домашнее задание \week}}

\medskip

\textbf{Дайте обоснованные ответы на следующие вопросы.}


\vspace{5mm}


\p В левой доле двудольного графа 300 вершин, в правой~"--- 400
вершин. Степени всех вершин в левой доле равны 4, а всех вершин в
правой доле равны 3. Докажите, что в таком графе есть
паросочетание размера 300. 

Воспользуемся теоремой Холла. Докажем, что $|S| \leq |G(S)| \quad \forall S \subseteq L$, где $L$ - левая доля.
Дадим оценку на $|G(S)|$: $|G(S)| \geq \dfrac{4 * |S|}{3} \geq |S|$. 
Такая оценка получается при следующих рассуждениях. $|G(S)|$ минимально, когда в каждую вершину из $G(S)$ попало 3 ребра, исходящие из $S$.
Всего ребер исходящих из $S:$ $n = 4|S|$. Получаем минимальный $|G(S)| = \dfrac{4|S|}{3}$

\p В  неориентированном графе на 2024 вершинах (необязательно двудольном) между любыми тремя вершинами есть хотя бы два ребра. Докажите, что в графе есть совершенное паросочетание (из 1012 рёбер).

Пусть $P$ - паросочетание, которое после всех преобразований станет совершенным. 
Добавим в $P$ две вершины, между которыми есть ребро.
Возьмем любые две вершины, которые еще не входят в $P$. 
Докажем, что можно расширить $P$ с помощью новых вершин. 
Для этого выберем в $P$ любые две вершины соединенные ребром. 
Обозначим их $p_1, p_2$, а новые вершины $n_1, n_2$. 
Если между вершинами $n_1, n_2$ есть ребро, просто добавим их в $P$ (1).
Пусть между новыми вершинами нет ребра.
Рассмотрим вершины $p_1, p_2, n_1$.
Из условия следует, что существует либо ребро $(p_1, n_1)$, либо $(p_2, n_1)$. 
Так же существует одно из ребер $(p_1, n_2), (p_2, n_2)$. 
Если новые вершины "присоединены" к разным старым вершинам, 
то обновим $P$, заменим пару $(p_1, p_2)$ на две новые пары. 
Пусть новые вершины "присоединены" к одной и той же старой вершине, 
тогда рассмотрим вершины $n_1, n_2, x$, где $x -$ вершина, к которой не присоединены вершины $n_1, n_2$ (2). 
Все эти вершины образуют независимое множество (по 1 и 2), хотя между ними должно быть 2 ребра. 
Противоречие, значит такого случая не может быть.
Таким образом, мы расширили $P$ на 2 вершины.
Продолжим процесс, пока не расширим $P$ до совершенного паросочетания.


\p В неориентированном графе на 101 вершине есть независимое множества размера 52. Докажите, что в этом графе нет паросочетания размера 50. 

Рассмотрим независимое множество размера 52, назовем его $v$, чтобы вершина из него входила в паросочетание, 
она должна быть соединена с вершиной, не входящей в $v$.
Таких вершин 101 - 52 = 49. Получается, что таким способом можно набрать паросочетание размера $49$. 
Если добавлять в паросочетание пары вершин, обе из которых не входят в $v$, 
то мы лишь уменьшаем размер итогового паросочетания,
потому что вершины из $v$ теперь имеют меньше вариантов, с которыми их можно соединить.

\p В неориентированном графе на $n$ вершинах есть вершинное покрытие размера
10. Докажите, что в таком графе нет простого пути длины 21. (В простом
пути все вершины разные, длина пути~"--- количество рёбер в нём.)

Пусть существует путь размера 21.
Начнем выбирать вершины, чтобы покрыть данный путь, заметим, что каждая вершина покрывает не больше 2 ребер. 
Таким образом, чтобы покрыть все ребра пути нужно как минимум 11 вершин, но у нас есть вершинное покрытие из 10 - противоречие. 
\end{document}
